\documentclass[a4paper,12pt]{article}

\usepackage[utf8]{inputenc}
\usepackage[T1]{fontenc}
\usepackage{color}
\definecolor{grey}{rgb}{0.9,0.9,0.9}
\definecolor{teal}{rgb}{0.0,0.5,0.5}
\definecolor{violet}{rgb}{0.5,0,0.5}
\usepackage[margin=2.5cm]{geometry}
\usepackage{graphicx}
\usepackage[francais]{babel}
\usepackage[babel=true]{csquotes}
\usepackage{listings}

\title{TP2 - Programmation logique inductive}
\author{\textsc{Paul Chaignon} - \textsc{Ulysse Goarant}}
\date{\today}

\begin{document}
\lstset{language=Prolog}

\maketitle

\section{Prise en main - Les trains de Michalsky}


\section{Une affaire de famille}
Nos exemples positifs sont les suivants.

\begin{lstlisting}[frame=single]
fille_de(mary, ann).
fille_de(rosy, mary).
fille_de(eve, tom).
fille_de(lisa, tom).
\end{lstlisting}

Nos exemples négatifs sont les suivants. Nous avons choisi de les indiquer intégralement.

\begin{lstlisting}[frame=single]
fille_de(ann, mary).
fille_de(ann, ann).
fille_de(ann, tom).
fille_de(ann, lisa).
fille_de(ann, rosy).
fille_de(ann, eve).
fille_de(ann, bob).

fille_de(mary, mary).
fille_de(mary, tom).
fille_de(mary, lisa).
fille_de(mary, rosy).
fille_de(mary, eve).
fille_de(mary, bob).

fille_de(tom, mary).
fille_de(tom, ann).
fille_de(tom, tom).
fille_de(tom, lisa).
fille_de(tom, rosy).
fille_de(tom, eve).
fille_de(tom, bob).

fille_de(rosy, ann).
fille_de(rosy, tom).
fille_de(rosy, lisa).
fille_de(rosy, rosy).
fille_de(rosy, eve).
fille_de(rosy, bob).

fille_de(eve, mary).
fille_de(eve, ann).
fille_de(eve, lisa).
fille_de(eve, rosy).
fille_de(eve, eve).
fille_de(eve, bob).

fille_de(lisa, mary).
fille_de(lisa, ann).
fille_de(lisa, lisa).
fille_de(lisa, rosy).
fille_de(lisa, eve).
fille_de(lisa, bob).

fille_de(bob, mary).
fille_de(bob, ann).
fille_de(bob, tom).
fille_de(bob, lisa).
fille_de(bob, rosy).
fille_de(bob, eve).
fille_de(bob, bob).
\end{lstlisting}

Le \textit{background knowledge} est défini de la manière suivante.

\begin{lstlisting}[frame=single]
:- modeb(*, pere(+, -)).
:- modeb(1, pere(-, +)).
:- modeb(*, mere(+, -)).
:- modeb(1, mere(-, +)).
:- modeb(*, fille(-)).

:- determination(fille_de/2, pere/2).
:- determination(fille_de/2, mere/2).
:- determination(fille_de/2, fille/1).

pere(tom, eve).
pere(tom, lisa).
pere(tom, bob).
mere(ann, mary).
mere(mary, rosy).
fille(mary).
fille(eve).
fille(ann).
fille(rosy).
fille(lisa).
\end{lstlisting}

\section{Les figures du Poker}


\end{document}