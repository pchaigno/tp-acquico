\documentclass[a4paper,12pt]{article}

\usepackage[utf8]{inputenc}
\usepackage[T1]{fontenc}
\usepackage{color}
\definecolor{grey}{rgb}{0.9,0.9,0.9}
\definecolor{teal}{rgb}{0.0,0.5,0.5}
\definecolor{violet}{rgb}{0.5,0,0.5}
\usepackage[margin=2.5cm]{geometry}
\usepackage{graphicx}
\usepackage[francais]{babel}
\usepackage[babel=true]{csquotes}
\usepackage{listings}

\title{TP2 - Programmation logique inductive}
\author{\textsc{Paul Chaignon} - \textsc{Ulysse Goarant}}
\date{\today}

\begin{document}
\lstset{language=Prolog}

\maketitle

\section{Prise en main - Les trains de Michalsky}

2. L'opérateur de raffinement d'Aleph a une approche bottom-up.

\section{Une affaire de famille}
Nos exemples positifs sont les suivants.

\begin{lstlisting}[frame=single]
fille_de(mary, ann).
fille_de(rosy, mary).
fille_de(eve, tom).
fille_de(lisa, tom).
\end{lstlisting}

Nos exemples négatifs sont les suivants. Nous avons choisi de les indiquer intégralement.

\begin{lstlisting}[frame=single]
fille_de(ann, mary).
fille_de(ann, ann).
fille_de(ann, tom).
fille_de(ann, lisa).
fille_de(ann, rosy).
fille_de(ann, eve).
fille_de(ann, bob).

fille_de(mary, mary).
fille_de(mary, tom).
fille_de(mary, lisa).
fille_de(mary, rosy).
fille_de(mary, eve).
fille_de(mary, bob).

fille_de(tom, mary).
fille_de(tom, ann).
fille_de(tom, tom).
fille_de(tom, lisa).
fille_de(tom, rosy).
fille_de(tom, eve).
fille_de(tom, bob).

fille_de(rosy, ann).
fille_de(rosy, tom).
fille_de(rosy, lisa).
fille_de(rosy, rosy).
fille_de(rosy, eve).
fille_de(rosy, bob).

fille_de(eve, mary).
fille_de(eve, ann).
fille_de(eve, lisa).
fille_de(eve, rosy).
fille_de(eve, eve).
fille_de(eve, bob).

fille_de(lisa, mary).
fille_de(lisa, ann).
fille_de(lisa, lisa).
fille_de(lisa, rosy).
fille_de(lisa, eve).
fille_de(lisa, bob).

fille_de(bob, mary).
fille_de(bob, ann).
fille_de(bob, tom).
fille_de(bob, lisa).
fille_de(bob, rosy).
fille_de(bob, eve).
fille_de(bob, bob).
\end{lstlisting}

Le \textit{background knowledge} est défini de la manière suivante.

\begin{lstlisting}[frame=single]
:- modeb(*, pere(+, -)).
:- modeb(1, pere(-, +)).
:- modeb(*, mere(+, -)).
:- modeb(1, mere(-, +)).
:- modeb(*, fille(-)).

:- determination(fille_de/2, pere/2).
:- determination(fille_de/2, mere/2).
:- determination(fille_de/2, fille/1).

pere(tom, eve).
pere(tom, lisa).
pere(tom, bob).
mere(ann, mary).
mere(mary, rosy).
fille(mary).
fille(eve).
fille(ann).
fille(rosy).
fille(lisa).
\end{lstlisting}

Ci-dessous se trouve la réponse fournie par Aleph.
Il a en fait créé des règles spécifiques à chaque exemple positif et n'a donc pas réussi à généraliser le concept de \textit{fille de}.

\begin{lstlisting}[frame=single]
[Rule 1] [Pos cover = 1 Neg cover = 0]
fille_de(mary,ann).

[Rule 2] [Pos cover = 1 Neg cover = 0]
fille_de(rosy,mary).

[Rule 3] [Pos cover = 1 Neg cover = 0]
fille_de(eve,tom).

[Rule 4] [Pos cover = 1 Neg cover = 0]
fille_de(lisa,tom).
\end{lstlisting}

\section{Les figures du Poker}

La règle obtenue pour la détermination du carré est la suivante.

\begin{lstlisting}[frame=single]
[Rule 1] [Pos cover = 236 Neg cover = 0]
carre(A) :-
   cartes(A,main(B,C,D,E,F)), valeur(E,G), valeur(D,G), valeur(C,G), 
   valeur(B,G).
\end{lstlisting}

Même en augmentant le nombre d'exemples positifs et négatifs Aleph n'a pas réussi à déterminer une règle pour paire, brelan ou suite.
Ces figures sont plus difficiles à faire apprendre car le point commun des exemples positifs est plus complexe.

La suite impose d'avoir conscience de la relation d'ordre sur les valeurs que Aleph ne connait pas.
Les figures paire et brelan ne peuvent être des carrés.
Il faut donc comprendre que les paires, par exemple, ont uniquement deux valeurs identiques et pas plus.

\end{document}