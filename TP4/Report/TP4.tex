\documentclass[a4paper,12pt]{article}

\usepackage[utf8]{inputenc}
\usepackage[T1]{fontenc}
\usepackage{color}
\definecolor{grey}{rgb}{0.9,0.9,0.9}
\definecolor{teal}{rgb}{0.0,0.5,0.5}
\definecolor{violet}{rgb}{0.5,0,0.5}
\usepackage[margin=2.5cm]{geometry}
\usepackage{graphicx}
\usepackage[francais]{babel}
\usepackage[babel=true]{csquotes}
\usepackage{listings}

\title{TP4 - Génération de règles d'association}
\author{\textsc{Paul Chaignon} - \textsc{Ulysse Goarant}}
\date{\today}

\begin{document}

\maketitle

\section{Fouille de données sous Weka}

\subsubsection*{Question 1.1}

Weka détermine les itemsets fréquents (en indiquant leur nombre par taille) et les règles associées de confiance suffisante.


\subsubsection*{Question 1.2}

En fonction de la mesure, les règles générées varient.
Certaines sont cependant communes à des mesures différentes.


\subsubsection*{Question 1.3}

Règle : outlook=overcast ==> play=yes
Calcul de la confiance : N(outlook=overcast et play=yes)/N(outlook=overcast)=4/4=1
Calcul du lift : N(omega)*N(outlook=overcast et play=yes)/N(outlook=overcast)*N(play=yes)=(70*4)/(4*4)=17.5
Calcul du levier : P(outlook=overcast et play=yes) - P(outlook=overcast)*P(play=yes)=


\subsubsection*{Question 1.8}

Les règles obtenues selon les paramètres par défaut sont les suivantes :

 1.  marital-status=_Never-married  capital-gain=_faible-gain-placement  capital-loss=_faible-perte-placement 67 ==>  gain=_<=50K 64    conf:(0.96)
 2. age=_jeune  workclass=_Private  capital-gain=_faible-gain-placement 66 ==>  gain=_<=50K 63    conf:(0.95)
 3.  marital-status=_Never-married  capital-gain=_faible-gain-placement 74 ==>  gain=_<=50K 70    conf:(0.95)
 4.  workclass=_Private  education-num=_peu-eduque  capital-gain=_faible-gain-placement 73 ==>  gain=_<=50K 69    conf:(0.95)
 5. age=_jeune  workclass=_Private 70 ==>  gain=_<=50K 66    conf:(0.94)
 6.  workclass=_Private  education-num=_peu-eduque  capital-gain=_faible-gain-placement  capital-loss=_faible-perte-placement 67 ==>  gain=_<=50K 63    conf:(0.94)
 7.  workclass=_Private  education-num=_peu-eduque 76 ==>  gain=_<=50K 71    conf:(0.93)
 8. age=_jeune  capital-gain=_faible-gain-placement  native-country=_United-States 72 ==>  gain=_<=50K 67    conf:(0.93)
 9.  marital-status=_Never-married  capital-loss=_faible-perte-placement 71 ==>  gain=_<=50K 66    conf:(0.93)
10.  workclass=_Private  education-num=_peu-eduque  capital-loss=_faible-perte-placement 70 ==>  gain=_<=50K 65    conf:(0.93)

Elles sont toutes associées à l'attribut \textit{gain=_<=50K}.
Ces personnes sont caractérisées entre autres comme étant jeune, d'un niveau de formation modeste, et n'ayant peu ou pas de gains à l'aide de placements.


\subsubsection{Question 1.9}

L'option \textit{car} impose que l'attribut à droite dans les règles soit le gain ce qui facilite grande l'interprétation des règles puisque notre objectif était d'obtenir des informations relatives au gain.



\section{\'Etude de cas : articles de presse}

\subsubsection*{Question 2.1}

Le nombre de mots considérés joue un rôle prépondérant dans les calculs au moment de la formation des itemsets fréquents.


\subsubsection*{Question 2.2}

Dans notre modélisation, une transaction correspond à un article, les items correspondent aux mots.
Nous avons écrit un script pour former un fichier \textit{ARFF} à partir d'un ensemble d'articles (un article par ligne).
Les attributs correspondent à la présence ou l'abscence de chacun des mots précédents.
En appliquant la méthode Apriori, on espère former des règles qui nous indiquent le sujet d'un article en fonction des mots présents dans celui-ci.

\end{document}